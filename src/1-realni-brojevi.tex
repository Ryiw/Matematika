\chapter{Realni brojevi}\label{ch:realni-brojevi}

\section{Računanje s realnim brojevima}\label{sec:računanje-s-realnim-brojevima}

Skup je bilo koja kolekcija različitih objekata u cjelini.
Elementi skupa mogu biti raznih vrsta: brojevi, ljudi, slova abecede, drugi skupovi itd.
Skupovi se dogovorno označavaju velikim slovima $A$, $B$, $C$, $\ldots$ te vitičastim zagradama $\{\ \}$ u koje se upisuju elementi.
Skupovi se mogu definirati, tj. opisati riječima ili eksplicitnim nabrajanjem svih elemenata između vitičastih zagrada.
Skup može biti konačan, beskonačan ili prazan.

Ako je nešto element nekog pojedinačnog skupa, odnosno pripada skupu, tada koristimo oznaku $\in$, a u slučaju da nije element skupa odnosno ne pripada skupu oznaku $\notin$.

Skupove možemo uspoređivati, pa ukoliko su svi elementi skupa $A$ i $B$ isti, možemo reći da je skup $A$ jednak skupu $B$, a tvrdnju zapisujemo $A = B$.
Nisu li svi elementi skupa $A$ i $B$ isti možemo zaključiti da je skup $A$ različit od skupa $B$, a tvrdnju zapisujemo $A \neq B$.

Ako je svaki član skupa $A$ također član skupa $B$, tada se za skup $A$ kaže da je podskup skupa $B$, a zapisuje se $A \subseteq B$.
Može se, također, zapisati $B \supseteq A$ odnosno skup $B$ je nadskup skupa $A$.
Ako je skup $A$ podskup i nije jednak skupu $B$, tada se za skup $A$ kaže da je pravi podskup skupa $B$, a zapisuje se $A \subset B$ ili možemo reći da je skup $B$ pravi nadskup skupa $A$ i zapisati $B \supset A$.

Postoji nekoliko načina za konstruiranje novih skupova od već postojećih.
Dva se skupa mogu „zbrojiti“ i to nazivamo unija skupova.
Unija skupova $A$ i $B$, označena sa $A \cup B$, je skup svih elemenata koji su članovi ili skupa $A$ ili skupa $B$.

Novi se skup također može konstruirati određivanjem „zajedničkih“ elemenata obaju skupova.
To nazivamo presjek skupova.
Presjek skupova $A$ i $B$, označen sa $A \cap B$, je skup svih elemenata koji su članovi i skupa $A$ i skupa $B$.