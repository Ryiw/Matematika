\newpage

\subsubsection{Zadaci}

\begin{zadatak}
    Kolika je vrijednost broja $\displaystyle 17 + \frac{\sin (53^{\circ})}{\log_5 10}$ zaokružena na četiri decimale?
    \begin{tasks}(4)
		\task $17.5582$
		\task $17.5583$
		\task $17.2767$
		\task $17.2768$
	\end{tasks}
\end{zadatak}

\begin{zadatak}
    Kolika je vrijednost broja $\displaystyle \frac{\sqrt[4]{380}}{5-\sqrt {4}}$ zaokružena na tri decimale?
    \begin{tasks}(4)
		\task $-1.116$
		\task $-1.117$
		\task $1.471$
		\task $1.472$
	\end{tasks}
\end{zadatak}

\begin{zadatak}
    Koji od brojeva pripada skupu iracionalnih brojeva?
    \begin{tasks}(4)
		\task $1.55$
		\task $-\sqrt{36}$
		\task $-\displaystyle\frac{2}{5}$
		\task $-\sqrt{7}$
	\end{tasks}
\end{zadatak}

\begin{zadatak}
	Koji od brojeva pripada skupu racionalnih brojeva?
    \begin{tasks}(4)
		\task $3-\pi$
		\task $\sqrt{0.81}$
		\task $4.112123\ldots$
		\task $(\sqrt{2}-4)^2$
	\end{tasks}
\end{zadatak}

\begin{zadatak}
	Koja je od navedenih tvrdnji istinita?
    \begin{tasks}(4)
		\task $2.\dot{4} \in \mathbb{N}$
		\task $\sqrt{8} \in \mathbb{Q}$
		\task $\displaystyle -\frac{7}{4} \in \mathbb{R}$
		\task $\pi \in \mathbb{Z}$
	\end{tasks}
\end{zadatak}

\begin{zadatak}
	Koja je od navedenih tvrdnji istinita?
    \begin{tasks}(4)
		\task $\mathbb{R} \subset \mathbb{Z}$
		\task $\mathbb{Q} \subset \mathbb{Z}$
		\task $\mathbb{Z} \subset \mathbb{Q}$
		\task $\mathbb{Z} \subset \mathbb{N}$
	\end{tasks}
\end{zadatak}

\begin{zadatak}
	Zbroj dvaju brojeva različitih od nule jednak je 0.
	Ti brojevi su
    \begin{tasks}(4)
		\task prosti
		\task suprotni
		\task jednaki
		\task recipročni
	\end{tasks}
\end{zadatak}

\begin{zadatak}
	Umnožak dvaju brojeva različitih od nule jednak je 1.
	Ti brojevi su
    \begin{tasks}(4)
		\task prosti
		\task suprotni
		\task jednaki
		\task recipročni
	\end{tasks}
\end{zadatak}

\begin{zadatak}
	Za svaki cijeli broj $z$ broj $4z - 4$ je
    \begin{tasks}(4)
		\task pozitivan
		\task negativan
		\task paran
		\task neparan
	\end{tasks}
\end{zadatak}

\begin{zadatak}
	Podijelimo li broj $n$ sa 7 te pritom dobijemo ostatak 4, tada $n$ možemo prikazati u obliku
    \begin{tasks}(4)
		\task $n=7k$
		\task $n=7k+4$
		\task $n=4k+7$
		\task $n=\displaystyle \frac{4}{7}k$
	\end{tasks}
\end{zadatak}

\begin{zadatak}
	Koliko ima prirodnih brojeva $n$ za koje je razlomak $\displaystyle \frac{3n-6}{3n}$ prirodan broj?
    \begin{tasks}(4)
		\task 1
		\task 2
		\task 3
		\task 4
	\end{tasks}
\end{zadatak}

\pagebreak

\begin{zadatak}
	Koliko ima cijelih brojeva $z$ za koje je razlomak $\displaystyle \frac{4}{3z-2}$ cijeli broj?
    \begin{tasks}(4)
		\task 1
		\task 2
		\task 3
		\task 4
	\end{tasks}
\end{zadatak}

\begin{zadatak}
	Koliko ima cijelih brojeva $z$ za koje je razlomak $\displaystyle \frac{3z^2+2}{z^2-2}$ cijeli broj?
    \begin{tasks}(4)
		\task 1
		\task 2
		\task 3
		\task 4
	\end{tasks}
\end{zadatak}

\begin{zadatak}
	Računska operacija između dva cijela broja definirana je izrazom $\displaystyle a \otimes b = \frac{2a-b}{a-2b}$.
	Koliko je $5 \otimes 2 + 3 \otimes 1$?
    \begin{tasks}(4)
		\task 10
		\task 13
		\task 16
		\task 19
	\end{tasks}
\end{zadatak}

\begin{zadatak}
	Odredite broj između 7950 i 8150 koji podijeljen sa 174 ima količnik jednak ostatku.
\end{zadatak}

\begin{zadatak}
	Napišite neki prirodni broj koji je veći od 3183 i koji pri dijeljenju sa 7 daje ostatak 4.
\end{zadatak}

\begin{zadatak}
	Koliko ima prirodnih brojeva $n$ takvih da je $2 \leq \sqrt{n} < 3$?
    \begin{tasks}(4)
		\task 2
		\task 3
		\task 4
		\task 5
	\end{tasks}
\end{zadatak}

\begin{zadatak}
	Koliko ima cijelih brojeva $z$ takvih da je $z^2 < 10$?
    \begin{tasks}(4)
		\task 5
		\task 6
		\task 7
		\task 8
	\end{tasks}
\end{zadatak}

\begin{zadatak}
	Nazivnik razlomka je broj 7.
	Koji prirodan broj $n$ je brojnik razlomka, ako je razlomak veći od $\displaystyle \frac{1}{4}$ i manji od $\displaystyle \frac{1}{3}$?
    \begin{tasks}(4)
		\task 1
		\task 2
		\task 3
		\task 4
	\end{tasks}
\end{zadatak}

\begin{zadatak}
	Odredite najmanji prirodan broj $n$ koji je djeljiv sa 72 i sa 189.
\end{zadatak}

\begin{zadatak}
	Pretvorite 4 dana 11 sati 7 minuta i 24 sekunde u minute.
\end{zadatak}

\begin{zadatak}
	Koliko je vremena prošlo od 21.\ srpnja 2022.\ godine u 12 sati i 35 minuta do 23.\ srpnja 2022.\ godine u 10 sati i 20 minuta?
    \begin{tasks}(4)
		\task 45 h i 15 min
		\task 45 h i 45 min
		\task 46 h i 15 min
		\task 46 h i 45 min
	\end{tasks}
\end{zadatak}

\begin{zadatak}
	Mjera jednog kuta u trokutu iznosi $\displaystyle \frac{5\pi}{9}$ radijana.
	Kolika iznosi ta mjera izražena u stupnjevima?
    \begin{tasks}(4)
		\task 40°
		\task 60°
		\task 80°
		\task 100°
	\end{tasks}
\end{zadatak}

\pagebreak

\begin{zadatak}
	Koliko je $4.15$ dm$^2$ izraženo u mm$^2$?
    \begin{tasks}(4)
		\task $41.5$ mm$^2$
		\task $415$ mm$^2$
		\task $4150$ mm$^2$
		\task $41500$ mm$^2$
	\end{tasks}
\end{zadatak}

\begin{zadatak}
	Gustoća bakra iznosi 8.96 $\displaystyle \frac{\text{g}}{\text{cm}^3}$.
	Koliko iznosi gustoća izražena u $\displaystyle \frac{\text{kg}}{\text{m}^3}$?
    \begin{tasks}(4)
		\task 0.896 $\displaystyle \frac{\text{kg}}{\text{m}^3}$
		\task 89.6 $\displaystyle \frac{\text{kg}}{\text{m}^3}$
		\task 8960 $\displaystyle \frac{\text{kg}}{\text{m}^3}$
		\task 896000 $\displaystyle \frac{\text{kg}}{\text{m}^3}$
	\end{tasks}
\end{zadatak}

\begin{zadatak}
	U jednoj šumi hrast pokriva površinu od 18 hektara, a bukva površinu od 7350 četvornih hvati.
	Kolika je ukupna površina te šume izražena u metrima kvadratnim?
	Napomena: 1 hektar = 2780 četvornih hvati = 10000 $\text{m}^2$.
\end{zadatak}

\begin{zadatak}
	Spremnik za vodu se sastoji od dva dijela.
	U prvi dio stane 7 imperijalnih galona, a u  drugi 35 litara vode.
	Kolika je ukupna zapremina spremnika izražena u metrima kubnim?
	Napomena: 1 imperijalni galon = 4.54609 litara, a 1~litra = 1~$\text{dm}^2$.
\end{zadatak}

\newpage